\documentclass[11pt]{article}
\usepackage[utf8]{inputenc}
\usepackage[slovene]{babel}
\usepackage{relsize}
\usepackage{amsthm}
\usepackage{amsmath, amssymb, amsfonts}
\usepackage{bbm}

\newcommand{\p}{\mathbb{P}}
\newcommand{\E}{\mathbb{E}}
\newcommand{\R}{\mathbb{R}}
\newcommand{\1}{\mathbbm{1}}
\newcommand{\n}{\mathcal{N}}
\newcommand{\B}{\mathcal{B}}
\newcommand{\N}{\mathbb{N}}
\newcommand{\C}{\mathcal{C}}
\newcommand{\F}{\mathcal{F}}
\newcommand{\h}{\mathcal{H}}
\newcommand{\A}{\mathcal{A}}

\theoremstyle{definition}
\newtheorem{definicija}{Definicija}[section]

\theoremstyle{definition}
\newtheorem{trditev}{Trditev}[section]

\theoremstyle{definition}
\newtheorem{izrek}{Izrek}[section]

\newtheorem*{posledica}{Posledica}
\newtheorem*{opomba}{Opomba}
\newtheorem*{komentar}{Komentar}
\newtheorem{lema}{Lema}
\newtheorem*{dokaz}{Dokaz}
\newtheorem*{posplošitev}{Posplošitev}
\newtheorem{metoda}{Metoda}

\title{Verjetnost in statistika - definicije, trditve in izreki}
\author{Oskar Vavtar \\
po predavanjih profesorja Jaka Smrekarja}
\date{2020/21}

\begin{document}
\maketitle
\pagebreak
\tableofcontents
\pagebreak

% #################################################################################################

\section{SLUČAJNI VEKTORJI}
\vspace{0.5cm}

\begin{definicija}[Komulativna porazdelitvena funkcija]

Slučajni vektor je taka funkcija/preslikava $\vec{X} = (X_1, \ldots, X_n): \Omega \rightarrow \mathbb{R}^n$, kjer je $\Omega$ verjetnostni prostor, za katero so množice
\begin{align*}
\{ X_1 \leq x_1, \ldots, X_n \leq x_n \} ~&=~ \{ X_1 \in (-\infty, x_1], \ldots, X_n \in (-\infty, x_n] \} \\
&=~ \{ \vec{X} \in (-\infty, x_1] \times \ldots \times (-\infty, x_n] \} \\
&=~ {\vec{X}}^{-1} \left( (-\infty, x_1] \times \ldots \times (-\infty, x_n] \right)
\end{align*}
dogodki za vse \textit{realne} $n$-terice $(x_1, \ldots, x_n) \in \mathbb{R}^n$. \\

\noindent \textit{Komulativna porazdelitvena funkcija} slučajnega vektorja $\vec{X}$ je funkcija $F_X: \mathbb{R}^n \rightarrow [0, 1]$ s predpisom
$$F_{\vec{X}}(x_1, \ldots, x_n) ~=~ \mathbb{P}(X_1 \leq x_1, \ldots, X_n \leq x_n).$$

\end{definicija}
\vspace{0.5cm}

\begin{trditev}[Lastnosti KPF]
~
\begin{enumerate}

	\item[(1)] \begin{align*}
		\lim_{x_i \rightarrow -\infty} F_{\vec{X}}(x_1, \ldots, x_n) ~&=~ 0 \\
		\lim_{\substack{x_1 \rightarrow \infty \\ \cdots \\ x_n \rightarrow \infty}} F_{\vec{X}}(x_1, \ldots, x_n) ~&=~ 1 
		\end{align*}
	
	\item[(2)] \textsc{Monotonost}: \\Če je $x_i \leq y_i$ za $\forall i \in \{ 1,\ldots, n \}$, je
	$$F_{\vec{X}}(x_1, \ldots, x_n) ~=~ F_{\vec{X}}(\vec{x}) ~\leq~ F_{\vec{X}}(\vec{y}).$$
	To sledi iz monotonosti $\mathbb{P}$.
	
	\item[(3)] \textsc{Zveznost z desne}:
	$$\lim_{\vec{y} \searrow \vec{x}} F_{\vec{X}}(\vec{y}) ~=~ F_{\vec{X}}(\vec{x})$$
	Tu $\vec{y} \searrow \vec{x}$ interpretiramo kot $\vec{y_i} \searrow \vec{x_i}$ za $\forall i$.
	
\end{enumerate}
\end{trditev}
\vspace{0.5cm}

\begin{opomba}

Lastnosti (1), (2) in (3) karakterizirajo družino abstraktnih komulativnih porazdelitvenih funkcij v primeru slučajnih spremeljivk ($n=1$). V večrazsežnem prostoru to ne drži.

\end{opomba}
\vspace{0.5cm}

\begin{izrek}

Če je $F: \mathbb{R}^2 \rightarrow [0, 1]$ zadošča (1), (2), (3) in (4): 
$$F(b, d) - F(a, d) - F(b, c) + F(a, c) ~\geq~ 0$$
za vse četverice $a < b$ in $c < d$, je $F$ \textit{komulativna porazdelitvena funkcija} nekega slučajnega vektorja $(X, Y): \Omega \rightarrow \mathbb{R}^2$.

\end{izrek}
\vspace{0.5cm}

\begin{definicija}[``Zvezni'' slučajni vektorj]

Slučajni vektor $\vec{X}: \Omega \rightarrow \mathbb{R}^n$ ima (\textit{zvezno}) gostoto, če obstaja taka \textit{zvezna} funkcija $f_{\vec{X}}: \mathbb{R}^n \rightarrow [0, \infty)$, da zanjo velja
$$\mathbb{P}(\vec{x} \in \mathcal{B}) ~=~ \int_{\mathcal{B}} f_{\vec{X}}(x_1, \ldots, x_n)\,dx_1 \ldots dx_n,$$ 
za vsako Borelovo množico $\mathcal{B} \subset \mathbb{R}^n$.

\end{definicija}
\vspace{0.5cm}

\begin{posplošitev}

Pravimo, da ima vektor $\vec{X} = (X_1, \ldots, X_n)$ $n$-razsežno normalno porazdelitev s parametrom $\vec{\mu} \in \mathbb{R}$ in $\Sigma \in \mathbb{R}^{n \times n}$ (\textit{simetrična} in \textit{pozitivno definitna}), če ima gostoto:
$$\mathlarger{f_{\vec{X}}(\vec{x}) ~=~ (2\pi)^{-\frac{n}{2}} (\det{\Sigma})^{-1} e^{-\frac{1}{2} \langle~ \Sigma^{-1}(\vec{x} - \vec{\mu})~, ~(\vec{x} - \vec{\mu}) ~\rangle}}$$ 

\end{posplošitev}
\vspace{0.5cm}

\pagebreak

% #################################################################################################

\section{NEODVISNOST}
\vspace{0.5cm}

\begin{definicija}

Komponente $X_1, \ldots, X_n$ slučajnega vektorja $\vec{X} = (X_1, \ldots, X_n)$ so \textit{neodvisne}, če velja
$$F_{\vec{X}}(x_1, \ldots, x_n) ~=~ F_{X_1}(x_1) \cdot \ldots \cdot F_{X_n}(x_n)$$
za vse $n$-terice $(x_1, \ldots, x_n) \in \mathbb{R}^n$.

\end{definicija}
\vspace{0.5cm}

\begin{opomba}

Enakost lahko prepišemo v 
$$\mathbb{P} \left( \vec{X} \in (-\infty, x_1] \times \ldots \times (-\infty, x_n] \right) ~=~ \mathbb{P} \left( X_1 \in (-\infty, x_1] \right) \cdot \ldots \cdot \mathbb{P} \left( X_n \in (-\infty, x_n] \right)$$

\end{opomba}
\vspace{0.5cm}

\begin{trditev}

Naj ima $(X, Y)$ \textit{``zvezno''} gostoto $f(X, Y)$ in naj bosta $f_X$ in $f_Y$ robni gostoti. Tedaj sta $X$ in $Y$ \textit{neodvisni} natanko takrat, ko 
$$f_{(X, Y)}(x, y) ~=~ f_X(x) \cdot f_Y(y)$$
za skoraj vse pare $x$ in $y$.

\end{trditev}
\vspace{0.5cm}

\begin{trditev}[Posledica prejšnje trditve]

Naj ima $(X, Y)$ \textit{``zvezno''} gostoto $f_{(X, Y)}(x, y)$. Tedaj sta $X$ in $Y$ neodvisni natanko takrat, ko velja
$$f_{(X, Y)}(x, y) ~=~ \Phi(x) \Psi(y)$$
za skoraj vse $x$ in $y$, za neki \textit{nenegativni} integralski funkciji.

\end{trditev}
\vspace{0.5cm}

\pagebreak

% #################################################################################################

\section{FUNKCIJE in TRANSFORMACIJE \\slučajnih spremenljivk in vektorjev}
\vspace{0.5cm}

% *************************************************************************************************

\subsection{Slučajne spremenljivke}
\vspace{0.5cm}

\begin{trditev}[Diskretni primer]

Če je $X$ \textit{diskretna slučajna spremenljivka} z vrednostmi $\{ x_i \mid i \in I \}$ in je $g$ funkcija, ki preslika množico $\{ x_i \mid i \in I \}$ na množico $\{ y_j \mid j \in J \}$, je $g \circ X = g(X)$ \textit{diskretna slučajna spremenljivka} z verjetnostno funkcijo
$$\p(g(X) = y_j) ~=~ \sum_{i:~g(x_i)=y_j} \p(X = x_i)$$

\end{trditev}
\vspace{0.5cm}

\begin{trditev}[Zvezni primer]

Naj ima slučajna spremenljivka $X$ \textit{``zvezno''} gostoto $f_X$, ki je različna od $0$ natanko na intervalu $(a, b)$, kjer $-\infty \leq a \leq b \leq \infty$. 

Naj bo $g: (a, b) \rightarrow (c, d)$ \textit{zvezna bijekcija}. Zanima nas funkcija $g(X)$ slučajne spremenljivke $X$.

Velja tudi
$$\p(X \in (a, b)) ~=~ \int_a^b f_X (x)\,dx ~=~ \int_{\R} f_X (x)\,dx ~=~ 1.$$

Velja $\{ g(X) \leq z \} = \{ X \leq g^{-1}(z) \}$. Ker je $X$ slučajna spremenljivka, so  $\{ X \leq g^{-1}(z) \}$ dogodki za $z \in (c, d).$ Sledi, da je $g(X)$ slučajna spremenljivka in velja:
$$F_{g(X)}(z) ~=~ \p(g(X) \leq z) ~=~ \begin{cases}
1~~~~~~~~~~~~~~; ~~~z \geq d \\
F_X(g^{-1}(z))~; ~~~z \in (c, d) \\
0~~~~~~~~~~~~~~; ~~~z \leq c
\end{cases}$$
Če je $g$ \textit{odvedljiva}, sledi:
$$f_{g(X)}(z) ~=~ \frac{d}{dz} F_{g(X)}(z)  ~=~ \begin{cases}
\frac{f_X(g^{-1}(z))}{g'(g^{-1}(z))}~; ~~~z \in (c, d) \\
0~~~~~~~~~~~~~; ~\text{sicer}
\end{cases}$$
Za splošno \textit{odvedljivo bijekcijo} $g: (a, b) \rightarrow (c, d)$ velja t.i. \textit{transformacijska formula}:
$$f_{g(X)}(z) ~=~ \frac{f_X(g^{-1}(z))}{|g'(g^{-1}(z))|}, ~~~\text{za}~ z \in (c, d).$$
 
\end{trditev}
\vspace{0.5cm}

% *************************************************************************************************

\subsection{Slučajni vektorji}
\vspace{0.5cm}

\begin{trditev}

Naj bo $\vec{X}: \Omega \rightarrow \R^n$ slučajni vektor in naj bo $h: \R^n \rightarrow \R$ \textit{zvezna} funkcija. Tedaj je $h(\vec{X}) = h(X_1, \ldots, X_n)$ \textit{slučajna spremenljivka}.

\end{trditev}
\vspace{0.5cm}

\begin{trditev}

Naj bo $\vec{X}: \Omega \rightarrow \R^n$ slučajni vektor z gostoto $f_{\vec{X}}$. Dalje naj bo $g: \R^n \rightarrow \R^n$ (ali $g: D \rightarrow E$ za primerni množici $D, E \subset \R^n$, kjer $\p(\vec{X} \in D) = 1$) \textit{zvezno diferenciabilna bijekcija}. Tedaj ima slučajni vektor $g(\vec{X})$ gostoto
$$f_{g(\vec{X})}(\vec{z}) ~=~ f_{\vec{X}} (g^{-1}(\vec{z})) \cdot |\det{Jg^{-1}(\vec{z})|}$$
v točkah $\vec{z} \in \R^n$ (oz. $\vec{z} \in E$).

Za dvorazsežne vektorje, kjer je $\vec{z} = (u, v)$ in $(U, V) = g(X, Y)$ ter posledično $(X, Y) = g^{-1}(U, V)$ se transformacijska formula glasi
$$f_{(U, V)}(u, v) ~=~ f_{(X ,Y)}(g^{-1}(u, v)) \cdot |\det{Jg^{-1}(u, v)}|$$
oziroma
$$f_{(U, V)}(u, v) ~=~ f_{(X, Y)}(x(u, v), y(u, v)) \cdot \left| \begin{vmatrix}
\frac{\partial x}{\partial u} & \frac{\partial x}{\partial v} \\
\frac{\partial y}{\partial u} & \frac{\partial y}{\partial v}
\end{vmatrix} \right|.$$
Upoštevaje $g \circ g^{-1} = \text{id}$ in posledično
$$Jg(g^{-1}(u, v)) \cdot Jg^{-1}(u, v) ~=~ I$$
zgornje prepišemo v
$$f_{(U, V)}(u, v) ~=~ \frac{f_{(X, Y))}(g^{-1}(u, v))}{|\det{Jg(g^{-1}(u, v))}|} ~=~ \frac{f_{(X, Y)}(x(u, v), y(u, v))}{\left| \begin{vmatrix}
\frac{\partial u}{\partial x} & \frac{\partial u}{\partial y} \\
\frac{\partial v}{\partial x} & \frac{\partial v}{\partial y}
\end{vmatrix} \Big|_{(x(u, v),~y(u, v))} \right|}.$$ 

\end{trditev}
\vspace{0.5cm}

\begin{trditev}

Naj za $\vec{X}: \Omega \rightarrow \R^n$ velja $\vec{X} \sim N(\vec{\mu}, \Sigma)$ in naj bo $A: \R^n \rightarrow \R^n$ \textit{obrnljiva} matrika. Dalje naj bo $\vec{\nu} \in \R^n$. Tedaj
$$A \vec{X} + \vec{\nu} \sim \n(A \vec{\mu} + \vec{\nu}, A \Sigma A^T).$$

\end{trditev}
\vspace{0.5cm}

% *************************************************************************************************

\pagebreak

% #################################################################################################

\section{PRIČAKOVANA VREDNOST \\zveznih slučajnih spremenljivk in vektorjev}
\vspace{0.5cm}

\begin{definicija}

Naj bo $X$ slučajna spremenljivka z ``zvezno'' gostoto $f_X: \R \rightarrow [0, \infty)$. \textit{Pričakovana vrednost} slučajne spremenljivke $X$ je
$$\E(X) ~=~ \int_{-\infty}^{\infty} x \cdot f_X\,dx,$$
če integral absolutno konvergira. \\

Za slučajni vektor $\vec{X} = (X_1, \ldots, X_n)$ definiramo pričakovano vrednost po komponentah
$$\E(\vec{X}) ~=~ (\E(X_1), \ldots, \E(X_2)) \in \R^n.$$
Podobno definiramo tudi pričakovano vrednost slučajnih matrik.

\end{definicija}
\vspace{0.5cm}

% *************************************************************************************************

\subsection{Pričakovana vrednost funkcij slučajnih spremenljivk in vektorjev}
\vspace{0.5cm}

\begin{trditev}

Naj bo $g: \R \rightarrow \R$ (ali $g:(a, b) \rightarrow (c, d)$) zvezna funkcija. Tedaj je
$$\E(g(X)) ~=~ \int_{-\infty}^{\infty} g(x) f_X(x)\,dx,$$
če obstaja.

\end{trditev}
\vspace{0.5cm}

\begin{posledica}

Za $p \neq 0$ velja:
$$\E(|x|^p) ~=~ \int_0^{\infty} |z|^p (f_X(z) + f_X(-z))\,dz ~=~ \ldots ~=~ \int_{-\infty}^{\infty} |x|^p f_X(x)\,dx.$$

\end{posledica}
\vspace{0.5cm}

\begin{trditev}

Naj bo $h: \R^2 \rightarrow \R$ (oziroma $h: D \rightarrow \R$ v primeru, da $\p((X, Y) \in D) = 1)$ \textit{zvezna} funkcija in $(X, Y)$ slučajni vektor z gostoto $f_{(X, Y)}$. Tedaj velja
$$\E(h(X, Y)) ~=~ \int_{R^2 ~(ali~D)} h(x, y) f_{(X, Y)}(x, y)\,dx\,dy,$$
če obstaja.

\end{trditev}
\vspace{0.5cm}

\begin{posledica}

$$\E(XY) ~=~ \int_{\R} x \cdot y \cdot f_{(X, Y)}(x, y)\,dx\,dy,$$
če integral absolutno konvergira. 

\end{posledica}
\vspace{0.5cm}

\begin{opomba}

Če sta $X$ in $Y$ \textit{neodvisni} slučajni spremenljivki, za kateri obstajata $\E(X)$ in $\E(Y)$, potem obstaja $\E(XY)$ in je
$$\E(XY) ~=~ \E(X) \cdot \E(Y).$$

\end{opomba}
\vspace{0.5cm}

\begin{posledica}[Aditivnosti pričakovane vrednosti]

Če je $X \leq Y$ (z verjetnostjo $1$), velja
$$\E(X) ~\leq~ \E(Y)$$
če $\E(|Y|) < \infty)$.

\end{posledica}
\vspace{0.5cm}

\begin{izrek}

Naj za slučajni spremenljivki $X$ in $Y$ velja $\E(X^2) < \infty$ in $\E(Y^2) < \infty$. Tedaj obstaja $\E(XY)$ in velja
$$\E(|XY|) ~\leq~ \sqrt{\E(X^2)\E(Y^2)}.$$
$$\text{Enakost nastopi} ~\iff~ \frac{Y}{\sqrt{\E(Y^2)}} = \pm \frac{X}{\sqrt{\E(X^2)}}~\text{skoraj gotovo.}$$

\end{izrek}
\vspace{0.5cm}

\begin{posledica}

Če je $\E(X^2) < \infty$, je $\E(|X|) < \infty$ in
$$\E(|X|) ~\leq~ \sqrt{\E(X^2)}.$$

\end{posledica}
\vspace{0.5cm}

% *************************************************************************************************

\subsection{Disperzija, kovarianca in variančno-kovariančno matrika}
\vspace{0.5cm}

\begin{definicija}

\textit{Disperzija (razpršenost, varianca)} slučajne spremenljivke $X$ je 
$$\E((X - \E(X))^2),$$
če ta pričakovana vrednost obstaja. Ta obstaja, če obstaja $\E(X^2)$ in tedaj velja
$$Var(X) ~=~ \E(X^2) - \E(X)^2.$$

\end{definicija}
\vspace{0.5cm}

\begin{definicija}

\textit{Standardni odklon} slučajne spremenljivke $X$ definiramo kot
$$\sigma(X) ~=~ \sigma_X ~=~ \sqrt{Var(X)}.$$
Pripomnimo, da ima $\sigma(X)$ iste enote kot $X$ in je zato $\sigma(X)$ ``količina'' ki je neposredno primerljiva z $X$.

\end{definicija}
\vspace{0.5cm}

\begin{trditev}[Lastnosti variance]

Naj za slučajno spremenljivko $X$ obstaja $\E(X^2)$ (torej obstaja tudi $Var(X)$):
\begin{itemize}
	\item $Var(X) \geq 0$ in $Var(X) = 0$ $\iff$ $X \equiv \E(X)$ skoraj gotovo,
	
	\item $Var(X)$ je minimum funkcije $a \mapsto \E((X-a)^2)$,
	
	\item $Var(aX+b) = a^2 Var(X)$,
	
	\item $Var(X+Y) = Var(X) + Var(Y)$, če sta $X$ in $Y$ \textit{neodvisni}.
\end{itemize}

\end{trditev}
\vspace{0.5cm}

\begin{komentar}

Vemo, da je $\sigma^2$ v $\n(\mu, \sigma^2)$ disperzija.

\end{komentar}
\vspace{0.5cm}

\begin{opomba}

Za \textit{zvezno} slučajno spremenljivko $X$ z gostoto $f_X$ je 
$$Var(X) ~=~ \int_{-\infty}^{\infty}(x - \E(X))^2 f_X(x)\,dx.$$

\end{opomba}
\vspace{0.5cm}

\begin{definicija}

\textit{Kovarianca} slučajnih spremenljivk $X$ in $Y$ je
$$Cov(X, Y) ~=~ \E((X - \E(X))(Y - \E(Y))),$$
če obstaja. $Cov(X,Y)$ obstaja, če obstajajo $\E(X)$, $\E(Y)$, $\E(XY)$. Če je tako, velja
$$Cov(X, Y) ~=~ \E(XY) - \E(X)\E(Y).$$
Če sta $X$ in $Y$ neodvisni in imata pričakovano vrednost, potem 
$$Cov(X, Y) ~=~ 0.$$  

\end{definicija}
\vspace{0.5cm}

\begin{trditev}[Lastnosti kovariance]
~
\begin{itemize}
	\item $Cov(X, Y)$ obstaja, če obstajata $Var(X)$ in $Var(Y)$ ter velja \textit{Cauchy-Schwarzova neenakost}
	$$|Cov(X, Y)| ~\leq~ \sqrt{Var(X)Var(Y)} ~=~ \sigma(X)\sigma(Y),$$
	enakost nastopi $\iff$ $\frac{Y-\E(Y)}{\sigma(Y)} = \pm \frac{X-\E(X)}{\sigma(X)}$ skoraj gotovo. V tem primeru pravimo, da sta $X$ in $Y$ skoraj gotovo linearno povezani.
	
	\item $Cov(X, X) = Var(X)$
	
	\item Kovarianca je \textit{simetrična bilinearna} funkcija. Dovolj je preveriti bilinearnost v eni spremenljivki.
	
	\item Če ima $(X, Y)$ \textit{zvezno} gostoto $f_X$, potem je
	$$Cov(X, Y) ~=~ \iint_{\R^2} (x-\E(X))(y-\E(Y)) f_{(X, Y)}(x, y)\,dx\,dy.$$
	
	\item Če imata $X$ in $Y$ disperzijo, jo ima tudi $X+Y$ in velja
	$$Var(X+Y) ~=~ Var(X) + 2Cov(X, Y) + Var(Y).$$
	Posplošitev na disperzijo vrste $X_1 + \ldots + X_n$:
	$$Var(X_1 + \ldots + X_n) ~=~ \sum_{i=1}^n Var(X_i) ~+~ 2\sum_{1 \leq i < j \leq n} Cov(X_i, X_j)$$
\end{itemize}

\end{trditev}
\vspace{0.5cm}

\begin{definicija}

\textit{Variančno-kovariančna matrika} slučajnega vektorja $\vec{X} = (X_1, \ldots, X_n)$ je matrika $Var(\vec{X}) \in \R^{n \times n}$ z elementi
$$[Var(\vec{X})]_{ij} ~=~ Cov(X_i, X_j).$$
To je \textit{simetrična} matrika z diagonalnimi elementi $Var(X_1), \ldots, Var(X_n)$.

\end{definicija}
\vspace{0.5cm}

\begin{definicija}

\textit{Pearsonov korelacijski koeficient} slučajnih spremenljivk $X$ in $Y$ z disperzijo je
$$\rho(X, Y) ~=~ \frac{Cov(X, Y)}{\sigma(X)\sigma(Y)} \in [-1, 1].$$

\end{definicija}
\vspace{0.5cm}

\begin{definicija}

Slučajni spremenljivki $X$ in $Y$ sta \textit{nekorelirani} natanko takrat, ko
$$\E(XY) ~=~ \E(X)\E(Y).$$
Vidimo:
\begin{itemize}
	\item Če sta $X$ in $Y$ \textit{neodvisni} slučajni spremenljivki s pričakovano vrednostjo, sta \textit{nekorelirani}.
	
	\item Obrat v splošnem ne drži, drži pa v primeru dvorazsežne normalno porazdeljenega slučajnega vektorja.
	
	\item Če imata $X$ in $Y$ disperzijo, sta nekorelirani $\iff$ $\rho(X, Y) = 0$.
\end{itemize}

\end{definicija}
\vspace{0.5cm}

% *************************************************************************************************

\pagebreak

% #################################################################################################

\section{POGOJNE PORAZDELITVE in \\POGOJNA PRIČAKOVANA VREDNOST}
\vspace{0.5cm}

\begin{definicija}

Naj bo $\B \subseteq \R$ Borelova množica. Pogojna verjetnost, da $Y$ zavzame vrednost v $\B$ pri pogoju $X=x$ je
\begin{align*}
\p(Y \in \B \mid X=x) ~&=~ \lim_{h \searrow 0} \p(Y \in \B \mid X \in (x-h, x+h)) ~=~ \\
~&=~ \lim_{h \searrow 0} \frac{\p(Y \in \B,~ X \in (x-h, x+h))}{\p(X \in (x-h, x+h))},
\end{align*}
če ta limita obstaja.

\end{definicija}
\vspace{0.5cm}

\begin{trditev}

Naj ima $(X, Y)$ \textit{``zvezno''} gostoto $f_{(X, Y)}$ in naj bo $f_X(x)$ \textit{zvezna} v $x$. Tedaj je
$$\p(Y \in \B \mid X=x) ~=~ \int_\B \frac{f_{(X, Y)}(x, y)}{f_X(x)}\,dy.$$

\end{trditev}
\vspace{0.5cm}

\begin{opomba}

Če sta $X, Y$ \textit{neodvisni}, je 
$$f_{(Y \mid X)}(y \mid x) ~=~ f_Y(y).$$
Brez privzetka \textit{zveznosti} je 
$$\p(Y \in \B \mid X=x) ~=~ \p(Y = \B).$$

\end{opomba}
\vspace{0.5cm}

\begin{trditev}

Velja tako imenovani \textit{zakon popolne verjetnosti}:
$$\p(Y \in \B) ~=~ \int_{-\infty}^\infty \p(Y \in \B \mid X=x)f_X(x)\,dx.$$

\end{trditev}
\vspace{0.5cm}

\begin{definicija}

\textit{Pogojna pričakovana vrednost} $Y$, pogojna na $X=x$, je pričakovana vrednost pogojne porazdelitve $(Y \mid X=x)$. V primeru, ko ima $(X, Y)$ \textit{``zvezno''} gostoto, to pomeni
$$\E(Y \mid X=x) ~=~ \int_{-\infty}^\infty y \cdot f_{(Y \mid X)}(y \mid x)\,dy.$$

\end{definicija}
\vspace{0.5cm}

\begin{trditev}

Zakon \textit{popolne pričakovane vrednosti}:
$$\E(Y) ~=~ \int_{-\infty}^\infty \E(Y \mid X=x) f_X(x)\,dx.$$

\end{trditev}
\vspace{0.5cm}

\begin{komentar}

Velja tudi:
\begin{align*}
\E(g(Y) \mid X=x) ~&=~ \int_{-\infty}^\infty g(y) f_{(Y \mid X)}(y \mid x)\,dy \\
\text{in}~~~~~~~~\E(g(Y)) ~&=~ \int_{-\infty}^\infty \E(g(y) \mid X=x) f_X(x)\,dx.
\end{align*}

\end{komentar}
\vspace{0.5cm}

\begin{definicija}

Pogojna pričakovana vrednost $\E(Y \mid X)$ je slučajna spremenljivka, definirana\footnote{Na istem verjetnostnem prostoru.} kot:
$$\E(Y \mid X)(\omega) ~=~ \E(Y \mid X = X(\omega)).$$
Če pišemo $g(x) = \E(Y \mid X=x)$, je $\E(Y \mid X) = g(X)$ \textit{transformacija} slučajne spremenljivke $X$. \\

Sledi:
\begin{align*}
\E(\E(Y \mid X)) ~&=~ \E(g(X)) ~=~ \int_{-\infty}^\infty g(y) f_X(x)\,dx ~=~ \\
~&=~ \int_{-\infty}^\infty \E(Y \mid X=x) f_X(x)\,dx ~=~ \E(Y).
\end{align*}
Pravimo, da je $\E(Y \mid X)$ najboljši približek za $Y$, če poznamo $X$.

\end{definicija}
\vspace{0.5cm}

\pagebreak

% #################################################################################################

\section{MOMENTI in \\MOMENTNO-RODOVNA FUNKCIJA}
\vspace{0.5cm}

\begin{definicija}

Naj bo $X$ slučajna spremenljivka in naj bosta $k \in \N$ in $a \in \R$. Moment slučajne spremenljivke $X$ glede na $a$ je
$$m_k(a) ~=~ \E((X-a)^k),$$
če obstaja, torej če $\E(|X-a|^k) < \infty$. \\

V zveznem primeru je
$$m_k(a) ~=~ \int_{-\infty}^\infty (x-a)^k f(x)\,dx,$$
če integral absolutno konvergira. \\

Pravimo:
\begin{itemize}
	\item $m_k(0)$ \ldots $k$-ti \textit{začetni} moment
	\item $m_k(\E(k))$ \ldots $k$-ti \textit{centralni} moment
\end{itemize}

\end{definicija}
\vspace{0.5cm}

\begin{trditev}

Če obstaja $m_n(a)$, potem obstajajo $m_k(a)$ za $k < n$.

\end{trditev}
\vspace{0.5cm}

\begin{trditev}

Če obstaja $m_n(a)$ za neki $a \in \R$, obstaja $m_n(b)$ za $b \in \R$.

\end{trditev}
\vspace{0.5cm}

\begin{posledica}

Če $m_n$ obstaja, je 
$$m_n(b) ~=~ \sum_{k=0}^\infty \binom{n}{k} (a-b)^{n-k} m_n(a).$$

\end{posledica}
\vspace{0.5cm}

\pagebreak
\begin{izrek}

Naj obstajajo vsi momenti $m_k = \E(X^k)$ in naj vrsta
$$\sum_{k=0}^\infty \frac{1}{k!} t^k \E(X^k)$$
absolutno konvergira za neki $t>0$. Potem je porazdelitev slučajne spremenljivke $X$ enolično določena z momenti. Če ima $Y$ enako lastnost in velja
$$\E(X^k) ~=~ \E(Y^k) ~~~\forall k,$$
sta $X$ in $Y$ \textit{enako porazdeljeni}.

\end{izrek}
\vspace{0.5cm}

\begin{komentar}

To pomeni, da je komulativna porazdelitvena funkcija $F_X: \R \rightarrow [0, 1]$ določena s \textit{števnim} naborom števil.

\end{komentar}
\vspace{0.5cm}

\begin{definicija}

\textit{Momentno-rodovna funkcija} slučajne spremenljivke $X$ je funkcija
$$M_X(t) ~=~ \E(e^{tX}),$$
ki je definirana za tista realna števila $t \in \R$, za katera pričakovana vrednost obstaja. Vedno je
$$M_X(0) ~=~ \E(1) ~=~ 1.$$
Za $t \neq 0$ je $y(x) = e^{tx}$ \textit{zvezno odvedljiva} bijekcija $\R \rightarrow (0, \infty)$ in za slučajno spremenljivko $X$ z gostoto $f_X$ velja
$$M_X(t) ~=~ \sum_{k=0}^\infty \frac{1}{k!} t^k \E(X^k).$$
To pomeni, da je $M_X$ \textit{analitična}; porazdelitev slučajne spremenljivke $X$ je \textit{enolično} določena z momenti.

\end{definicija}
\vspace{0.5cm}

\begin{trditev}

$$M_{aX+b}(t) ~=~ e^{tb} M_X(at)$$

\end{trditev}
\vspace{0.5cm}

\begin{posledica}

$$M_{\n(\mu, \sigma^2)}(t) ~=~ M_{\sigma \cdot \n(0, 1) + \mu}(t) ~=~ e^{t\mu} \cdot e^{\frac{(\sigma t)^2}{2}}$$

\end{posledica}
\vspace{0.5cm}

\begin{trditev}

Če sta slučajni spremenljivki $X$ in $Y$ \textit{neodvisni}, velja
$$M_{X+Y}(t) ~=~ M_X(t) \cdot M_Y(t).$$

\end{trditev}
\vspace{0.5cm}

\begin{lema}

Če sta slučajni spremenljivki $X$ in $Y$ \textit{neodvisni} in $f, g: \R \rightarrow \R$ (ali $f: (a, b) \rightarrow \R$, $g: (c, d) \rightarrow \R$) \textit{zvezni} funkciji, potem sta $f(X)$ in $g(Y)$ neodvisni:
\begin{align*}
\p(f(X) \in \B, ~g(Y) \in \C) ~&=~ \p(X \in f^{-1}(\B), ~Y \in g^{-1}(\C)) ~=~ \\
~&=~ \p(X \in f^{-1}(\B)) \cdot \p(Y \in g^{-1}(\C)),
\end{align*}
kjer smo pri zadnjem enačaju upoštevali \textit{neodvisnost}.

\end{lema}
\vspace{0.5cm}

\pagebreak

% #################################################################################################

\section{LIMITNI IZREKI}
\vspace{0.5cm}

\begin{izrek}[Krepki zakon velikih števil]

Verjetnost tistih vzorcev \\$s = (\omega_1, \omega_2, \omega_3, \ldots) \in S$, za katere je
$$\lim_{n \rightarrow \infty} \frac{X(\omega_1) + \ldots + X(\omega_n)}{n} ~=~ \lim_{n \rightarrow \infty} \frac{X_1(s) + \ldots + X_n(s)}{n} ~=~ \E(x),$$
je enaka $1$.

\end{izrek}
\vspace{0.5cm}

\begin{trditev}[Markova neenakost]

Naj bo $X$ slučajna spremenljivka s pričakovano vrednostjo in $a>0$. Tedaj je
$$\p(|X| > a) ~\leq~ \frac{1}{|a|} \E(|X|).$$

\end{trditev}
\vspace{0.5cm}

\begin{posledica}[Čebiševa neenakost]

Naj bo $\E(X^2) < \infty$ in bo $\varepsilon > 0$. Tedaj je
$$\p(|X - \E(X)| > \varepsilon) ~=~ \frac{Var(X)}{\varepsilon^2}.$$

\end{posledica}
\vspace{0.5cm}

\begin{izrek}[Šibki zakon velikih števil Markova]

Naj bodo $X_1, X_2, \ldots$ \textit{neodvisne} in \textit{enako porazdeljene} sličajne spremenljivka z varianco $\sigma^2 < \infty$ in pričakovano vrednostjo $\mu$. Tedaj $\forall \varepsilon > 0$ velja
$$\lim_{n \rightarrow \infty} \p \left( \left| \frac{X_1 + \ldots + X_n}{n} - \mu \right| > \varepsilon \right) ~=~ 0.$$

\end{izrek}
\vspace{0.5cm}

\begin{definicija}

Naj bodo $Y, Y_1, Y_2, Y_3, \ldots$ slučajne spremenljivke, definirane na skupnem verjetnostnem prostoru. Pravimo, da zaporedje $\{Y_n\}_n$ \textit{verjetnostno konvergira} k $Y$, pišemo $Y_n \xrightarrow[n \rightarrow \infty]{p} Y$, če $\forall \varepsilon > 0$ velja
$$\lim_{n \rightarrow \infty} \p(|Y_n - Y| > \varepsilon) ~=~ 0.$$

\end{definicija}
\vspace{0.5cm}

\begin{komentar}

Če so $X_1, X_2, \ldots$ \textit{neodvisne enako porazdeljene} slučajne spremenljivke s pričakovano vrednostjo $\mu$ in disperzijo $\sigma^2$, potem
$$\overline{X} ~=~ \frac{1}{n} \sum_{i=1}^n X_i ~\xrightarrow[n \rightarrow \infty]{p}~ \mu.$$
$\forall \varepsilon > 0:$
$$\lim_{n \rightarrow \infty} \p(|\overline{X} - \mu| > \varepsilon) ~=~ 0.$$

\end{komentar}
\vspace{0.5cm}

\begin{definicija}

Zaporedje $\{Y_n\}_n$ konvergira \textit{skoraj gotovo} k $Y$, $Y_n \xrightarrow[n \rightarrow \infty]{s.g.} Y$, če je
$$\p(\lim_{n \rightarrow} Y_n = Y) ~=~ \p(\{ s \mid \exists \lim_{n \rightarrow \infty} Y_n(s) = Y(s) \}) ~=~ 1.$$ 

\end{definicija}
\vspace{0.5cm}

\begin{izrek}[Krepki zakon velikih števil Kolmogorova]

Naj bodo $X_1, X_2, \ldots$ \textit{neodvisne} in  \textit{enako porazdeljene} slučajne spremenljivke s pričakovano vrednostjo $\mu$. Tedaj velja
$$\overline{X} ~=~ \frac{1}{n} \sum_{i=1}^n X_i ~\xrightarrow[n \rightarrow \infty]{s.g.}~ \mu.$$

\end{izrek}
\vspace{0.5cm}

\begin{komentar}
\begin{itemize}
	\item obstoj variance ni potreben
	\item za $X: \Omega \rightarrow \R$ smo konstruirali $X_i: \Omega^{\N} \rightarrow \R$
\end{itemize}
\end{komentar}
\vspace{0.5cm}

\begin{trditev}

Iz \textit{skoraj gotove} konvergence sledi konvergenca v \textit{verjetnosti}.

\end{trditev}
\vspace{0.5cm}

\begin{trditev}

Naj bo $g: \R \rightarrow \R$ \textit{zvezna} funkcija:
\begin{enumerate}

	\item[(1)] Če velja $Y_n \xrightarrow[n \rightarrow \infty]{s.g.} Y_n$, potem $g(Y_n) \xrightarrow[n \rightarrow \infty]{s.g.} g(Y)$.
	
	\item[(2)] Če velja $Y_n \xrightarrow[n \rightarrow \infty]{p} Y_n$, potem $g(Y_n) \xrightarrow[n \rightarrow \infty]{p} g(Y)$.

\end{enumerate}

\end{trditev}
\vspace{0.5cm}

\begin{izrek}[Centralni limitni izrek]

Naj bo $X_1, X_2, \ldots$ zaporedje \textit{neodvisnih} in \textit{enako porazdeljenih} slučajnih spremenljivk s pričakovano vrednostjo $\mu$ in varianco $\sigma^2 < \infty$. Tedaj $\forall x \in \R$:
$$\lim_{n \rightarrow \infty} \p \left( \frac{\frac{X_1 + \ldots + X_n}{n} - \mu}{\frac{\sigma}{\sqrt{n}}} \leq x \right) ~=~ \Phi(x) ~=~ \p(\n(0, 1) \leq x).$$

\end{izrek}
\vspace{0.5cm}

\begin{komentar}

$\frac{\frac{X_1 + \ldots + X_n}{n} - \mu}{\frac{\sigma}{\sqrt{n}}} = \frac{\overline{X} - \mu}{\frac{\sigma}{\sqrt{n}}}$ je \textit{standardizacija} od $\overline{X}$. Izjava \textit{centralnega limitnega izreka} pravi, da \textit{komulativne porazdelitvene funkcije} standardiziranih vzorčnih povprečij po točkah konvergirajo k komulativni porazdelitveni funkciji $\Phi$. Za realni števili $a<b$ sledi
$$\lim_{n \rightarrow \infty} \p \left( \frac{\overline{X} - \mu}{\frac{\sigma}{\sqrt{n}}} \in (a, b] \right) ~=~ \p \left( \overline{X} \in (\underbrace{\mu + a \frac{\sigma}{\sqrt{n}}}_{c}, \underbrace{\mu + b \frac{\sigma}{\sqrt{n}}}_{d}] \right).$$
Torej:
$$\p(\overline{X} \in (c, d]) ~\approx~ \Phi \left( \frac{d - \mu}{\frac{\sigma}{\sqrt{n}}} \right) ~-~ \Phi \left( \frac{c - \mu}{\frac{\sigma}{\sqrt{n}}} \right) ~=~ F_{\n(\mu, ~\frac{\sigma^2}{n})}(d) ~-~ F_{\n(\mu, ~\frac{\sigma^2}{n})}(c)$$
oziroma
$$\p(\overline{X} \in (c, d]) ~\approx~ \p \left(\n \left( \mu, \frac{\sigma^2}{n} \right) \in (c, d] \right) ~~~\text{za velike}~n.$$
Rečemo tudi
$$\overline{X} ~\dot{\sim}~ \n \left( \mu, \frac{\sigma^2}{n} \right).$$

\end{komentar}
\vspace{0.5cm}

\begin{izrek}[Izrek o zveznosti]

Če za slučajne spremenljivke $Y_1, Y_2, \ldots$, ki imajo momentno-rodovne funkcije na nekem intervalu $(-\delta, \delta)$, $\delta>0$, velja
$$\lim_{n \rightarrow \infty} M_{Y_n}(t) ~=~ e^{\frac{t^2}{2}} ~=~ M_{\n(0, 1)}(t) ~~~\forall t \in (-\delta, \delta),$$
potem $\forall x \in \R$ velja
$$\lim_{n \rightarrow \infty} \p(Y_n \leq x) ~=~ \Phi(x).$$

\end{izrek}
\vspace{0.5cm}

\pagebreak

% #################################################################################################

\section{TOČKOVNO OCENJEVANJE}
\vspace{0.5cm}

\begin{definicija}

Naj bo $c$ realnoštevilska ``karakteristika'' proučevane \hbox{porazdelitve}\footnote{npr. $c = \E(X)$, $c = \text{Var}(X)$, $c = \p(X \geq 2)$, \ldots}. Cenilka za $c$ je funkcija slučajnega vzorca $T = T(X_1, \ldots, X_n)$, s katero ocenjujemo $c$. Cenilka je določena s funkcijo $T: \R^n \rightarrow \R$; za \hbox{$T(X_1, \ldots, X_n) = \overline{X}$} je
$$T(x_1, \ldots, x_n) ~=~ \frac{1}{n}(x_1 + \ldots + x_n).$$
Cenilka $T$ karakteristike $c$ je \textit{nepristranska}, če velja
$$\E(T(X_1, \ldots, X_n)) ~=~ c(X)$$
za vsako dopustno porazdelitev. Natančneje: za vsako dopustno \hbox{porazdelitev} (k.p.f.) $F \in \F$ in vsak vzorec $X_1, \ldots, X_n \overset{NEP}{\sim} F$ velja
$$\E(T(X_1, \ldots, X_n)) ~=~ c(F).$$

\end{definicija}
\vspace{0.5cm}

\begin{definicija}

Naj bosta $U$ in $V$ \textit{nepristranski} cenilki za $c(X)$ v modelu $\F$. Tedaj ima $U$ enakomerno manjšo varianco od $V$, če velja
$$\text{Var}(U(X_1, \ldots, X_n)) ~\leq~ \text{Var}(V(X_1, \ldots, X_n))$$
$\forall F \in \F$ in vsak vzorec $X_1, \ldots, X_n \overset{NEP}{\sim} F$.

\end{definicija}
\vspace{0.5cm}

\begin{definicija}

Zaporedje cenilk $T_n$ za karakteristike $c$, je \textit{dosledno} (\textit{angl. consistent}), če $\forall F \in \F$ in vsako neskončno zaporedje $X_1, X_2, \ldots \overset{NEP}{\sim} F$ zaporedje $T_n(X_1, \ldots, X_n)$ konvergira h konstanti $c(F)$ verjetnostno. 
\end{definicija}
\vspace{0.5cm}

\begin{komentar}

Če $X_1, X_2, \ldots$ generiramo kot \textit{neodvisne} replikacije slučajne spremenljivke $X:\Omega \rightarrow \R$, lahko $X_i$ razumemo kot funkcijo na prostoru neskončnih vzorcev
$$S ~=~ \Omega^\N ~=~ \{\text{neskončna zaporedje iz $\Omega$}\}.$$

\end{komentar}
\vspace{0.5cm}

% *************************************************************************************************

\subsection{Metode za konstrukcijo cenilk}
\vspace{0.5cm}

\begin{metoda}[Metoda momentov]

Zanima nas karakteristika $c(X)$ proučevane slučajne spremenljivke $X: \Omega \rightarrow \R$ v nekem modelu. Če je $c(X)$ mogoče izraziti z momenti kot $c(X) = g(m_1, m_2, \ldots, m_2)$, potem $c(X)$ ocenjujemo s cenilko $\hat{c} = g(\hat{m}_1, \hat{m}_2, \ldots, \hat{m}_r)$. Če je $g$ \textit{zvezna}, je cenilka $\hat{c}$ \textit{dosledna}. V praksi imamo najpogosteje parametrični model z vektorskim parametrom $\vartheta = (\vartheta_1, \ldots, \vartheta_r)$, kjer izrazimo 
\begin{align*}
m_1 ~&=~ m_1(\vartheta_1, \ldots, \vartheta_r) \\
&\vdots \\
m_r ~&=~ m_r(\vartheta_1, \ldots, \vartheta_r) 
\end{align*}
Privzemimo, da znamo zgornji sistem razrešiti na $\vartheta$ z vektorsko funkcijo $(\vartheta_1, \ldots, \vartheta_r) = (g_1(m_1, \ldots, m_r), \ldots, g_r(m_1, \ldots, m_r))$. Potem za $\vartheta$ vzamemo cenilko
$$\hat{\vartheta} ~=~ (g_1(\hat{m}_1, \ldots, \hat{m}_r), \ldots, g_r(\hat{m}_1, \ldots, \hat{m}_r)).$$
Že vemo: če je $g = (g_1, \ldots, g_r)$ \textit{zvezna}, je $\hat{\vartheta}$ \textit{dosledna}.

\end{metoda}
\vspace{0.5cm}

\begin{metoda}[Metoda največjega verjetja]

Privzemimo parametrični model s prostorom parametrov $\Theta \subset \R^r$ in splošnim parametrom \hbox{$\vartheta = (\vartheta_1, \ldots, \vartheta_r) \in \Theta$}. Dalje privzemimo, da imajo dopustne porazdelitve verjetnostne funkcije (v diskretnem modelu) oz. gostote (v zveznem modelu) oblike
$$f(x;~\vartheta) ~=~ f(x;~\vartheta_1, \ldots, \vartheta_r).$$

\end{metoda}
\vspace{0.5cm}

\begin{definicija}

\textit{Funkcija verjetja} za vzorec velikosti $n$ je \hbox{$L: \R^n \times \Theta \rightarrow [0, \infty)$},
$$L(x_1, \ldots, x_n;~\vartheta) ~=~ f(x_1;~\vartheta) \cdot \ldots \cdot f(x_n;~\vartheta).$$
Kot funkcija vzorca $\vec{x} = (x_1, \ldots, x_n)$ je $L$ gostota vektorja $(x_1, \ldots, x_n)$. V teoriji verjetja $L$ pri danem $\vec{x}$ gledamo kot funkcijo parametra $\vartheta$. Oceno za $\vartheta$ pri danem $\vec{x}$ po metodi najmanjših kvadratov je tak $\hat{\vartheta} \in \Theta$, pri katerem ima $L(\vec{x};~\vartheta)$ maksimum, torej velja:
$$L(\vec{x};~\hat{\vartheta}) ~=~ \max_{\vartheta \in \Theta}{L(\vec{x};~\vartheta)}.$$
Oceno $\hat{\vartheta}$ analitično tipično izračunamo z odvajanjem. Zaradi preprostosti raje odvajamo logaritemsko funkcijo verjetja:
$$\ln(L(\vec{x};~\vartheta)) ~=~ \sum_{i=1}^n \ln{f(x_i;~\vartheta)}.$$
Stacionarne točke funkcije $\ln L$ imenujemo rešitve \textit{enačb verjetja} (EV)
$$\frac{\partial}{\partial \vartheta_j} (\ln L) ~=~ 0, ~~~1 \leq j \leq n.$$

\end{definicija}
\vspace{0.5cm}

\begin{definicija}[Cenilka največjega verjetja]

Če za vse možne (v diskretnem primeru) ali skoraj vse (v zveznem primeru) realizacije $\vec{x}$ slučajnega vektorja $\vec{X}$ obstaja ocena $\hat{\vartheta}(\vec{x})$ za $\vartheta$ po metodi največjega verjetja, to je
$$\max_{\vartheta \in \overline{\Theta}}{L(\vec{x};~\vartheta} ~=~ L(\vec{x};~\hat{\vartheta}(\vec{x})),$$
potem funkciji $\hat{\vartheta}: \R^n \rightarrow \overline{\Theta}$ pravimo \textit{cenilka največjega verjetja} (CNV) za $\vartheta$.

\end{definicija}
\vspace{0.5cm}

\begin{opomba}

Imamo parametrični model s parametričnim prostorom $\Theta$.
\begin{itemize}
	
	\item Množica ``možnih'' vrednosti se ne spreminja s $\vartheta$.
	
	\item Za ``skoraj vse vrednosti'' se ne spreminja s $\vartheta$: če je $D_{\hat{\vartheta}}$ definicijsko območje, mora veljati
	$$\p((X_1, \ldots, X_n) \in D_{\hat{\vartheta})} ~=~ 1$$
	za vsako dopustno porazdelitev, $\forall \vartheta$ in $\forall X_i \overset{NEP}{\sim} \vartheta$.	
	
\end{itemize}

\end{opomba}
\vspace{0.5cm}

\begin{izrek}

Naj bodo gostote (verjetnostne funkcije) $f(\vec{x};~\vartheta)$ \textit{dvakrat zvezno parcialno odvedljive} na $\vartheta_j$, naj bo $\{ \vec{x} \mid f(\vec{x};~\vartheta) > 0 \}$ \textit{neodvisna} od $\vartheta$ in naj veljajo še dodatni blagi regularnostni privzetki. ČE imajo (EV) enolične rešitve za vse dovoolj velike vzorce, ki jih označimo $\hat{\vartheta} = \hat{\vartheta}(x_1, \ldots, x_n)$, potem je $\hat{\vartheta}(x_1, \ldots, x_n)$ dosledno zaporedje cenilk za $\vartheta$.

\end{izrek}
\vspace{0.5cm}

% *************************************************************************************************

\pagebreak

% #################################################################################################

\section{INTERVALSKO OCENJEVANJE}
\vspace{0.5cm}

\begin{definicija}

Naj bo $c = c(X) = c(F_X)$ proučevana karakteristika slučajne spremenljivke $X$. Naj bo $\alpha \in (0, 1)$ vnaprej podano (``majhno'')  in naj bo $n$ velikost vzorca. Interval zaupanja za $c$ stopnje zaupanja $1 - \alpha$ je prireditev $\vec{X} = (X_1, \ldots, X_n) \mapsto [L(X_1, \ldots, X_n), U(X_1, \ldots, X_n)]$ za katero velja:
$$\p([L(X_1, \ldots, X_n), U(X_1, \ldots, X_n)] \ni c(F)) ~\geq~ 1 - \alpha$$
za vsako dopustno porazdelitev $F$ in $\forall X_1, \ldots, X_n \overset{NEP}{\sim} F$.

\end{definicija}
\vspace{0.5cm}

\begin{komentar}

Interval zaupanja je določen s funkcijama $L, U: \R^n \rightarrow \R$.

\end{komentar}
\vspace{0.5cm}

\begin{metoda}[Intervali zaupanja v normalnih modelih]
~\\
\begin{enumerate}

\item[\textbf{i)}] \textbf{Interval zaupanja za $\mu$ pri \textit{poznani} disperziji $\sigma^2$} 
$$X_1, \ldots, X_n \overset{NEP}{\sim} \n(\mu, \sigma^2) ~~~\Longrightarrow~~~ \frac{\overline{X} - \mu}{\frac{\sigma}{\sqrt{n}}} \sim \n(0, 1)$$
Privzemimo, da za $a<b$ ($a,b \in \R$) velja
$$\p(\n(0,1) \in [a, b]) ~=~ \Phi(b) - \Phi(a) ~=~ 1-\alpha.$$
Tedaj 
$$\p\left(a \leq \frac{\overline{X} - \mu}{\frac{\sigma}{\sqrt{n}}} \leq b\right) ~=~ 1 - \alpha ~~~\iff~~~ \p\left(\overline{X} - a\frac{\sigma}{\sqrt{n}} \geq \mu \geq \overline{X} - b\frac{\sigma}{\sqrt{n}}\right) ~=~ 1 - \alpha.$$
To pomeni, da je $[\overline{X} - b\frac{\sigma}{\sqrt{n}}, \overline{X} - a\frac{\sigma}{\sqrt{n}}]$ interval zaupanja za $\mu$ stopnje zaupanja $1 - \alpha$. Širina intervala znaša $(b-a)\frac{\sigma}{\sqrt{n}}$. Zaradi simetričnosti gostote $f_{\n(0, 1)}$, je minimum dosežen pri $a = -b$.
$$\Longrightarrow ~\Phi(b) ~=~ \Phi(b_{\text{sim}}) ~=~ 1 - \frac{\alpha}{2} ~~~\text{in}~~~ b ~=~ \Phi^{-1}\left(1-\frac{\alpha}{2}\right) ~=:~ z_{\frac{\alpha}{2}}$$
Standardni interval zaupanja je torej 
$$\left[\overline{X} - z_{\frac{\alpha}{2}} \frac{\sigma}{\sqrt{n}}, \overline{X} + z_{\frac{\alpha}{2}} \frac{\sigma}{\sqrt{n}}\right].$$

\item[\textbf{ii)}] \textbf{Interval zaupanja za $\mu$ z \textit{neznano} disperzijo $\sigma^2$} \\
\noindent Neznani $\sigma^2$ ocenimo s
$$S^2 ~=~ \frac{1}{n-1}\sum_{i=1}^n (X_i - \overline{X})^2.$$
Uporabimo Studentovo porazdelitev z $n-1$ prostorskimi stopnjami:
$$t_{n-1} ~=~ \frac{\overline{X} - \mu}{\frac{S}{\sqrt{n}}}.$$
Interval zaupanja za $\mu$ stopnje zaupanja $1-\alpha$
$$\left[\overline{X} - b\frac{S}{\sqrt{n}}, \overline{X} - a\frac{S}{\sqrt{n}}\right]$$
dobimo, če za $a<b$ velja
$$\p(t_{n-1} \in [a, b]) ~=~ F_{t_{n-1}}(b) - F_{t_{n-1}}(a) ~=~ 1-\alpha.$$
Ker je tudi $t_{n-1}$ simetrična porazdelitev, minimum dosežemo 
pri 
$$-a ~=~ b ~=~ F_{t_{n-1}}^{-1}(1-\frac{\alpha}{2}) ~=~ t_{n-1; \frac{\alpha}{2}}.$$
Standardni interval zaupanja:
$$\left[\overline{X} - t_{n-1; \frac{\alpha}{2}}\frac{S}{\sqrt{n}}, \overline{X} + t_{n-1; \frac{\alpha}{2}}\frac{S}{\sqrt{n}}\right].$$

\item[\textbf{iii)}] \textbf{Interval zaupanja za $\sigma^2$ z \textit{znano} pričakovano vrednostjo $\mu$} \\
\noindent Za $X_i \overset{NEP}{\sim} \n(\mu, \sigma^2)$ je 
$$\sum_{i=1}^n \left(\frac{X_i - \mu}{\sigma}\right)^2 ~\sim~ \chi_n^2.$$
Če velja $\p(\chi_n^2 \in [a, b]) = 1-\alpha$, je
$$\p\left(a \leq \frac{1}{\sigma^2}\sum_{i=1}^n(X_i - \mu)^2 \leq b\right) ~=~ 1 - \alpha$$
oziroma
$$\p\left(\left[\frac{1}{b}\sum_{i=1}^n(X_i - \mu)^2, \frac{1}{a}\sum_{i=1}^n(X_i - \mu)^2\right]\right) ~=~ 1 - \alpha.$$
Širina dobljenega intervala je $\left(\frac{1}{a}-\frac{1}{b}\right)\sum_{i=1}^n(X_i - \mu)^2$, z minimizacijo dobimo $\frac{1}{a}-\frac{1}{b}$ pri vezi
$$\p(\chi_n^2 \in [a, b]) ~=~ F_{\chi_n^2}(b) - F_{\chi_n^2}(a) ~=~ 1 - \alpha.$$

\end{enumerate}

\end{metoda}
\vspace{0.5cm}

\begin{izrek}[Student, Fisher]

Naj bodo $X_1, \ldots, X_n \overset{NEP}{\sim} \n(\mu, \sigma^2)$. Tedaj sta 
$$\overline{X} ~=~ \frac{1}{n}(X_1 + \ldots + X_n) ~~~\text{in}~~~ S^2 ~=~ \frac{1}{n-1} \sum_{i=1}^n (X_i - \overline{X})^2$$
neodvisni slučajni spremenljivki. Velja
$$\frac{n-1}{\sigma^2} \cdot S^2 ~=~ \sum_{i=1}^n \left( \frac{X_i - \overline{X}}{\sigma} \right)^2  ~\sim~ \chi_{n-1}^2 ~=~ \Gamma\left( \frac{n-1}{2}, \frac{1}{2} \right).$$

\end{izrek}
\vspace{0.5cm}

\begin{definicija}

Naj bosta $Z$ in $K$ \textit{neodvisni} slučajni spremenljivki in \hbox{$Z \sim \n(0, 1)$} ter $K \sim \chi_k^2$. Tedaj pravimo, da ima slučajna spremenljivka $\frac{Z}{\sqrt{\frac{K}{k}}}$ Studentovo porazdelitev s $k$ prostorskimi stopnjami. Označimo jo s $t_k$.

\end{definicija}
\vspace{0.5cm}

\begin{komentar}

Porazdelitev $t_k$ je zaradi \textit{neodvisnosti} funkcija porazdelitve slučajne spremenljivke $Z$ in slučajne spremenljivke $K$. Takoj sledi, da je $t_k$ \textit{simetrična} okrog $0$. Očitno sta $\frac{Z}{\sqrt{\frac{K}{k}}}$ in $\frac{-Z}{\sqrt{\frac{K}{k}}}$ enako porazdeljeni.

\end{komentar}
\vspace{0.5cm}

\begin{posledica}

Slučajna spremenljivka
$$t ~=~ \frac{\overline{X} - \mu}{\frac{S}{\sqrt{n}}} ~\sim~ t_{n-1}$$
(tu je $S = \sqrt{S^2}$).

\end{posledica}
\vspace{0.5cm}

\begin{metoda}[Naprej: Intervali zaupanja v normalnih modelih]
~\\
\begin{enumerate}

\item[\textbf{iv)}] \textbf{Interval zaupanja za $\sigma^2$ z \textit{neznano} pričakovano vrednostjo $\mu$} \\
\noindent Za $X_i \overset{NEP}{\sim} \n(\mu, \sigma^2)$ je 
$$\sum_{i=1}^n \left(\frac{X_i - \overline{X}}{\sigma}\right)^2 ~\sim~ \chi_{n-1}^2.$$
Če velja $\p(\chi_{n-1}^2 \in [a, b]) = 1-\alpha$, je
$$\p\left(a \leq \frac{1}{\sigma^2}\sum_{i=1}^n(X_i - \overline{X})^2 \leq b\right) ~=~ 1 - \alpha$$
oziroma
$$\p\left(\left[\frac{1}{b}\sum_{i=1}^n(X_i - \overline{X})^2, \frac{1}{a}\sum_{i=1}^n(X_i - \overline{X})^2\right]\right) ~=~ 1 - \alpha.$$
Širina dobljenega intervala je $\left(\frac{1}{a}-\frac{1}{b}\right)\sum_{i=1}^n(X_i - \overline{X})^2$, z minimizacijo dobimo $\frac{1}{a}-\frac{1}{b}$ pri vezi
$$\p(\chi_{n-1}^2 \in [a, b]) ~=~ F_{\chi_{n-1}^2}(b) - F_{\chi_{n-1}^2}(a) ~=~ 1 - \alpha.$$

\end{enumerate}

\end{metoda}
\vspace{0.5cm}

\pagebreak

% #################################################################################################

\section{PREIZKUŠANJE DOMNEV}
\vspace{0.5cm}

\begin{definicija}

Statistična domneva (hipoteza) je izjava, da porazdelitev slučajne spremenljivke $X$ pripada neki podmnožici $\h \subseteq \F$.

\end{definicija}
\vspace{0.5cm}

\begin{definicija}

Preizkus domneve $\h$ proti alternativi $\A$ za vzorec velikosti $n$ je odločitveno pravilo, ki na podlagi realizacije $n$ \textit{neodvisnih} replikacij proučevane slučajne spremenljivke odloči, ali $\h$ zavrnemo (in s tem sprejmemo $\A$) ali je ne zavrnemo (oz. jo ``sprejmemo'').

\end{definicija}
\vspace{0.5cm}

\begin{komentar}

Verjetnost napake 1. vrste je funkcija, definirana na domnevi $\h$. Če je $\h$ sestavljena iz ene porazdelitve (``enostavna''), je to v resnici eno število.

\end{komentar}
\vspace{0.5cm}

% #################################################################################################

\end{document}
>>>>>>> Stashed changes
